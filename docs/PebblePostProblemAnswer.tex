\documentclass[]{article}
\usepackage{lmodern}
\usepackage{amssymb,amsmath}
\usepackage{ifxetex,ifluatex}
\usepackage{fixltx2e} % provides \textsubscript
\ifnum 0\ifxetex 1\fi\ifluatex 1\fi=0 % if pdftex
  \usepackage[T1]{fontenc}
  \usepackage[utf8]{inputenc}
\else % if luatex or xelatex
  \ifxetex
    \usepackage{mathspec}
  \else
    \usepackage{fontspec}
  \fi
  \defaultfontfeatures{Ligatures=TeX,Scale=MatchLowercase}
\fi
% use upquote if available, for straight quotes in verbatim environments
\IfFileExists{upquote.sty}{\usepackage{upquote}}{}
% use microtype if available
\IfFileExists{microtype.sty}{%
\usepackage{microtype}
\UseMicrotypeSet[protrusion]{basicmath} % disable protrusion for tt fonts
}{}
\usepackage[margin=1in]{geometry}
\usepackage{hyperref}
\hypersetup{unicode=true,
            pdftitle={Pebble Post Problem Answer},
            pdfauthor={Xiaosheng Luo},
            pdfborder={0 0 0},
            breaklinks=true}
\urlstyle{same}  % don't use monospace font for urls
\usepackage{color}
\usepackage{fancyvrb}
\newcommand{\VerbBar}{|}
\newcommand{\VERB}{\Verb[commandchars=\\\{\}]}
\DefineVerbatimEnvironment{Highlighting}{Verbatim}{commandchars=\\\{\}}
% Add ',fontsize=\small' for more characters per line
\usepackage{framed}
\definecolor{shadecolor}{RGB}{248,248,248}
\newenvironment{Shaded}{\begin{snugshade}}{\end{snugshade}}
\newcommand{\KeywordTok}[1]{\textcolor[rgb]{0.13,0.29,0.53}{\textbf{#1}}}
\newcommand{\DataTypeTok}[1]{\textcolor[rgb]{0.13,0.29,0.53}{#1}}
\newcommand{\DecValTok}[1]{\textcolor[rgb]{0.00,0.00,0.81}{#1}}
\newcommand{\BaseNTok}[1]{\textcolor[rgb]{0.00,0.00,0.81}{#1}}
\newcommand{\FloatTok}[1]{\textcolor[rgb]{0.00,0.00,0.81}{#1}}
\newcommand{\ConstantTok}[1]{\textcolor[rgb]{0.00,0.00,0.00}{#1}}
\newcommand{\CharTok}[1]{\textcolor[rgb]{0.31,0.60,0.02}{#1}}
\newcommand{\SpecialCharTok}[1]{\textcolor[rgb]{0.00,0.00,0.00}{#1}}
\newcommand{\StringTok}[1]{\textcolor[rgb]{0.31,0.60,0.02}{#1}}
\newcommand{\VerbatimStringTok}[1]{\textcolor[rgb]{0.31,0.60,0.02}{#1}}
\newcommand{\SpecialStringTok}[1]{\textcolor[rgb]{0.31,0.60,0.02}{#1}}
\newcommand{\ImportTok}[1]{#1}
\newcommand{\CommentTok}[1]{\textcolor[rgb]{0.56,0.35,0.01}{\textit{#1}}}
\newcommand{\DocumentationTok}[1]{\textcolor[rgb]{0.56,0.35,0.01}{\textbf{\textit{#1}}}}
\newcommand{\AnnotationTok}[1]{\textcolor[rgb]{0.56,0.35,0.01}{\textbf{\textit{#1}}}}
\newcommand{\CommentVarTok}[1]{\textcolor[rgb]{0.56,0.35,0.01}{\textbf{\textit{#1}}}}
\newcommand{\OtherTok}[1]{\textcolor[rgb]{0.56,0.35,0.01}{#1}}
\newcommand{\FunctionTok}[1]{\textcolor[rgb]{0.00,0.00,0.00}{#1}}
\newcommand{\VariableTok}[1]{\textcolor[rgb]{0.00,0.00,0.00}{#1}}
\newcommand{\ControlFlowTok}[1]{\textcolor[rgb]{0.13,0.29,0.53}{\textbf{#1}}}
\newcommand{\OperatorTok}[1]{\textcolor[rgb]{0.81,0.36,0.00}{\textbf{#1}}}
\newcommand{\BuiltInTok}[1]{#1}
\newcommand{\ExtensionTok}[1]{#1}
\newcommand{\PreprocessorTok}[1]{\textcolor[rgb]{0.56,0.35,0.01}{\textit{#1}}}
\newcommand{\AttributeTok}[1]{\textcolor[rgb]{0.77,0.63,0.00}{#1}}
\newcommand{\RegionMarkerTok}[1]{#1}
\newcommand{\InformationTok}[1]{\textcolor[rgb]{0.56,0.35,0.01}{\textbf{\textit{#1}}}}
\newcommand{\WarningTok}[1]{\textcolor[rgb]{0.56,0.35,0.01}{\textbf{\textit{#1}}}}
\newcommand{\AlertTok}[1]{\textcolor[rgb]{0.94,0.16,0.16}{#1}}
\newcommand{\ErrorTok}[1]{\textcolor[rgb]{0.64,0.00,0.00}{\textbf{#1}}}
\newcommand{\NormalTok}[1]{#1}
\usepackage{longtable,booktabs}
\usepackage{graphicx,grffile}
\makeatletter
\def\maxwidth{\ifdim\Gin@nat@width>\linewidth\linewidth\else\Gin@nat@width\fi}
\def\maxheight{\ifdim\Gin@nat@height>\textheight\textheight\else\Gin@nat@height\fi}
\makeatother
% Scale images if necessary, so that they will not overflow the page
% margins by default, and it is still possible to overwrite the defaults
% using explicit options in \includegraphics[width, height, ...]{}
\setkeys{Gin}{width=\maxwidth,height=\maxheight,keepaspectratio}
\IfFileExists{parskip.sty}{%
\usepackage{parskip}
}{% else
\setlength{\parindent}{0pt}
\setlength{\parskip}{6pt plus 2pt minus 1pt}
}
\setlength{\emergencystretch}{3em}  % prevent overfull lines
\providecommand{\tightlist}{%
  \setlength{\itemsep}{0pt}\setlength{\parskip}{0pt}}
\setcounter{secnumdepth}{0}
% Redefines (sub)paragraphs to behave more like sections
\ifx\paragraph\undefined\else
\let\oldparagraph\paragraph
\renewcommand{\paragraph}[1]{\oldparagraph{#1}\mbox{}}
\fi
\ifx\subparagraph\undefined\else
\let\oldsubparagraph\subparagraph
\renewcommand{\subparagraph}[1]{\oldsubparagraph{#1}\mbox{}}
\fi

%%% Use protect on footnotes to avoid problems with footnotes in titles
\let\rmarkdownfootnote\footnote%
\def\footnote{\protect\rmarkdownfootnote}

%%% Change title format to be more compact
\usepackage{titling}

% Create subtitle command for use in maketitle
\newcommand{\subtitle}[1]{
  \posttitle{
    \begin{center}\large#1\end{center}
    }
}

\setlength{\droptitle}{-2em}

  \title{Pebble Post Problem Answer}
    \pretitle{\vspace{\droptitle}\centering\huge}
  \posttitle{\par}
    \author{Xiaosheng Luo}
    \preauthor{\centering\large\emph}
  \postauthor{\par}
      \predate{\centering\large\emph}
  \postdate{\par}
    \date{August 17, 2018}


\begin{document}
\maketitle

This RMarkdown document is for report purpose only. All codes can be
checked in
\href{https://github.com/trinityluo/PebblePostProblems.git}{github}.

\subsection{Read Data}\label{read-data}

\begin{enumerate}
\def\labelenumi{\arabic{enumi}.}
\item
  Manually download data from Google Drive into local disk. (An R
  package ``googledrive'' can automate this step)
\item
  Read csv files, then combine them.
\end{enumerate}

\begin{Shaded}
\begin{Highlighting}[]
\CommentTok{# obtain file names}
\NormalTok{filenames_cookie <-}\StringTok{ }\KeywordTok{dir}\NormalTok{(}\StringTok{'data/raw/cookie_match_sample'}\NormalTok{, }\DataTypeTok{full.names =}\NormalTok{ T)}
\NormalTok{filenames_event <-}\StringTok{ }\KeywordTok{dir}\NormalTok{(}\StringTok{'data/raw/event_sample'}\NormalTok{, }\DataTypeTok{full.names =}\NormalTok{ T)}

\CommentTok{# read csv files into one dataframe}
\NormalTok{cookie_match <-}\StringTok{ }\KeywordTok{do.call}\NormalTok{(rbind, }\KeywordTok{lapply}\NormalTok{(filenames_cookie, read.csv,}
                                       \DataTypeTok{stringsAsFactors =}\NormalTok{ F,}
                                       \DataTypeTok{colClasses =} \KeywordTok{c}\NormalTok{(}\StringTok{'character'}\NormalTok{, }\StringTok{'character'}\NormalTok{, }\StringTok{'Date'}\NormalTok{)))}

\NormalTok{event <-}\StringTok{ }\KeywordTok{do.call}\NormalTok{(rbind, }\KeywordTok{lapply}\NormalTok{(filenames_event, read.csv,}
                                \DataTypeTok{stringsAsFactors =}\NormalTok{ F,}
                                \DataTypeTok{colClasses =} \KeywordTok{c}\NormalTok{(}\KeywordTok{rep}\NormalTok{(}\StringTok{'character'}\NormalTok{, }\DecValTok{4}\NormalTok{), }\StringTok{'Date'}\NormalTok{)))}

\CommentTok{# save as R binary data files to fast read data again if something happens}
\KeywordTok{saveRDS}\NormalTok{(cookie_match, }\StringTok{'data/processed/cookie_match.Rds'}\NormalTok{)}
\KeywordTok{saveRDS}\NormalTok{(event, }\StringTok{'data/processed/event.Rds'}\NormalTok{)}
\CommentTok{# cookie_match <- readRDS('data/processed/cookie_match.Rds')}
\CommentTok{# event <- readRDS('data/processed/event.Rds')}
\end{Highlighting}
\end{Shaded}

\begin{enumerate}
\def\labelenumi{\arabic{enumi}.}
\setcounter{enumi}{2}
\tightlist
\item
  Merge data to find matched events
\end{enumerate}

\begin{Shaded}
\begin{Highlighting}[]
\CommentTok{# find all matched events}
\NormalTok{matched_events <-}\StringTok{ }\NormalTok{event }\OperatorTok\StringTok{ }
\StringTok{  }\KeywordTok{inner_join}\NormalTok{(cookie_match, }\DataTypeTok{by =} \KeywordTok{c}\NormalTok{(}\StringTok{'ppid'}\NormalTok{, }\StringTok{'date'}\NormalTok{))}
\end{Highlighting}
\end{Shaded}

\subsection{Question 1}\label{question-1}

\paragraph{What are the number of events for each brand for each
day?}\label{what-are-the-number-of-events-for-each-brand-for-each-day}

\begin{Shaded}
\begin{Highlighting}[]
\CommentTok{# number of events/brand/day}
\NormalTok{num_events <-}\StringTok{ }\NormalTok{event }\OperatorTok\StringTok{ }
\StringTok{  }\KeywordTok{count}\NormalTok{(brand_id, date)}

\KeywordTok{kable}\NormalTok{(num_events)}
\end{Highlighting}
\end{Shaded}

\begin{longtable}[]{@{}llr@{}}
\toprule
brand\_id & date & n\tabularnewline
\midrule
\endhead
1034 & 2018-03-15 & 1679\tabularnewline
1034 & 2018-03-16 & 1488\tabularnewline
1034 & 2018-03-17 & 1165\tabularnewline
1034 & 2018-03-18 & 1257\tabularnewline
1034 & 2018-03-19 & 1838\tabularnewline
1034 & 2018-03-20 & 1678\tabularnewline
1034 & 2018-03-21 & 1484\tabularnewline
1034 & 2018-03-22 & 1471\tabularnewline
1034 & 2018-03-23 & 1471\tabularnewline
1034 & 2018-03-24 & 1220\tabularnewline
1034 & 2018-03-25 & 1403\tabularnewline
1034 & 2018-03-26 & 1939\tabularnewline
1034 & 2018-03-27 & 1609\tabularnewline
1034 & 2018-03-28 & 1554\tabularnewline
1101 & 2018-03-15 & 12638\tabularnewline
1101 & 2018-03-16 & 12625\tabularnewline
1101 & 2018-03-17 & 12830\tabularnewline
1101 & 2018-03-18 & 14856\tabularnewline
1101 & 2018-03-19 & 14155\tabularnewline
1101 & 2018-03-20 & 13916\tabularnewline
1101 & 2018-03-21 & 13917\tabularnewline
1101 & 2018-03-22 & 13482\tabularnewline
1101 & 2018-03-23 & 13219\tabularnewline
1101 & 2018-03-24 & 12763\tabularnewline
1101 & 2018-03-25 & 14432\tabularnewline
1101 & 2018-03-26 & 13503\tabularnewline
1101 & 2018-03-27 & 12952\tabularnewline
1101 & 2018-03-28 & 12346\tabularnewline
1472 & 2018-03-15 & 82546\tabularnewline
1472 & 2018-03-16 & 82397\tabularnewline
1472 & 2018-03-17 & 82523\tabularnewline
1472 & 2018-03-18 & 95790\tabularnewline
1472 & 2018-03-19 & 97745\tabularnewline
1472 & 2018-03-20 & 92943\tabularnewline
1472 & 2018-03-21 & 93668\tabularnewline
1472 & 2018-03-22 & 85006\tabularnewline
1472 & 2018-03-23 & 78230\tabularnewline
1472 & 2018-03-24 & 72791\tabularnewline
1472 & 2018-03-25 & 86516\tabularnewline
1472 & 2018-03-26 & 83225\tabularnewline
1472 & 2018-03-27 & 74411\tabularnewline
1472 & 2018-03-28 & 73309\tabularnewline
\bottomrule
\end{longtable}

\paragraph{What are the number of matched events for each brand for each
day?}\label{what-are-the-number-of-matched-events-for-each-brand-for-each-day}

\begin{Shaded}
\begin{Highlighting}[]
\CommentTok{# number of Matched events/brand/day}
\NormalTok{num_match_events <-}\StringTok{ }\NormalTok{matched_events }\OperatorTok\StringTok{ }
\StringTok{  }\KeywordTok{inner_join}\NormalTok{(cookie_match, }\DataTypeTok{by =} \KeywordTok{c}\NormalTok{(}\StringTok{'ppid'}\NormalTok{, }\StringTok{'date'}\NormalTok{)) }\OperatorTok\StringTok{ }
\StringTok{  }\KeywordTok{count}\NormalTok{(brand_id, date)}

\KeywordTok{kable}\NormalTok{(num_match_events)}
\end{Highlighting}
\end{Shaded}

\begin{longtable}[]{@{}llr@{}}
\toprule
brand\_id & date & n\tabularnewline
\midrule
\endhead
1034 & 2018-03-15 & 25\tabularnewline
1034 & 2018-03-16 & 19\tabularnewline
1034 & 2018-03-17 & 6\tabularnewline
1034 & 2018-03-18 & 8\tabularnewline
1034 & 2018-03-19 & 14\tabularnewline
1034 & 2018-03-20 & 12\tabularnewline
1034 & 2018-03-21 & 85\tabularnewline
1034 & 2018-03-22 & 11\tabularnewline
1034 & 2018-03-23 & 77\tabularnewline
1034 & 2018-03-24 & 9\tabularnewline
1034 & 2018-03-25 & 6\tabularnewline
1034 & 2018-03-26 & 18\tabularnewline
1034 & 2018-03-27 & 8\tabularnewline
1034 & 2018-03-28 & 11\tabularnewline
1101 & 2018-03-15 & 469\tabularnewline
1101 & 2018-03-16 & 508\tabularnewline
1101 & 2018-03-17 & 493\tabularnewline
1101 & 2018-03-18 & 634\tabularnewline
1101 & 2018-03-19 & 452\tabularnewline
1101 & 2018-03-20 & 464\tabularnewline
1101 & 2018-03-21 & 478\tabularnewline
1101 & 2018-03-22 & 383\tabularnewline
1101 & 2018-03-23 & 412\tabularnewline
1101 & 2018-03-24 & 430\tabularnewline
1101 & 2018-03-25 & 557\tabularnewline
1101 & 2018-03-26 & 415\tabularnewline
1101 & 2018-03-27 & 457\tabularnewline
1101 & 2018-03-28 & 1638\tabularnewline
1472 & 2018-03-15 & 76917\tabularnewline
1472 & 2018-03-16 & 93598\tabularnewline
1472 & 2018-03-17 & 82869\tabularnewline
1472 & 2018-03-18 & 85015\tabularnewline
1472 & 2018-03-19 & 80829\tabularnewline
1472 & 2018-03-20 & 88260\tabularnewline
1472 & 2018-03-21 & 81874\tabularnewline
1472 & 2018-03-22 & 72562\tabularnewline
1472 & 2018-03-23 & 56429\tabularnewline
1472 & 2018-03-24 & 78506\tabularnewline
1472 & 2018-03-25 & 101238\tabularnewline
1472 & 2018-03-26 & 124461\tabularnewline
1472 & 2018-03-27 & 63626\tabularnewline
1472 & 2018-03-28 & 51861\tabularnewline
\bottomrule
\end{longtable}

\subsection{Question 2}\label{question-2}

\paragraph{What is the average number of events for each Day of Week for
each brand? Can you create a graph or plot to visualize this
information?}\label{what-is-the-average-number-of-events-for-each-day-of-week-for-each-brand-can-you-create-a-graph-or-plot-to-visualize-this-information}

\begin{Shaded}
\begin{Highlighting}[]
\CommentTok{# create new column for day of week}
\NormalTok{num_events <-}\StringTok{ }\NormalTok{num_events }\OperatorTok\StringTok{ }
\StringTok{  }\KeywordTok{mutate}\NormalTok{(}\DataTypeTok{day =} \KeywordTok{weekdays}\NormalTok{(date))}

\CommentTok{# calculate average events per day of week}
\NormalTok{avg_events_dayofweek <-}\StringTok{ }\NormalTok{num_events }\OperatorTok\StringTok{ }
\StringTok{  }\KeywordTok{group_by}\NormalTok{(brand_id, day) }\OperatorTok\StringTok{ }
\StringTok{  }\KeywordTok{summarise}\NormalTok{(}\DataTypeTok{avg_evnts =} \KeywordTok{mean}\NormalTok{(n))}

\CommentTok{# plot}
\NormalTok{week <-}\StringTok{ }\KeywordTok{c}\NormalTok{(}\StringTok{"Sunday"}\NormalTok{, }\StringTok{"Monday"}\NormalTok{, }\StringTok{"Tuesday"}\NormalTok{, }\StringTok{"Wednesday"}\NormalTok{, }\StringTok{"Thursday"}\NormalTok{, }\StringTok{"Friday"}\NormalTok{, }\StringTok{"Saturday"}\NormalTok{)}
\NormalTok{avg_events_dayofweek}\OperatorTok{$}\NormalTok{day <-}\StringTok{ }\KeywordTok{factor}\NormalTok{(avg_events_dayofweek}\OperatorTok{$}\NormalTok{day, }\DataTypeTok{levels =}\NormalTok{ week)}

\KeywordTok{ggplot}\NormalTok{(avg_events_dayofweek, }\KeywordTok{aes}\NormalTok{(}\DataTypeTok{x=}\NormalTok{day, }\DataTypeTok{y=}\NormalTok{avg_evnts, }\DataTypeTok{fill=}\NormalTok{brand_id)) }\OperatorTok{+}
\StringTok{  }\KeywordTok{geom_bar}\NormalTok{(}\DataTypeTok{stat=}\StringTok{'identity'}\NormalTok{, }\DataTypeTok{position=}\KeywordTok{position_dodge}\NormalTok{()) }\OperatorTok{+}
\StringTok{  }\KeywordTok{scale_fill_brewer}\NormalTok{(}\DataTypeTok{palette=}\StringTok{"Paired"}\NormalTok{) }\OperatorTok{+}
\StringTok{  }\KeywordTok{theme_minimal}\NormalTok{() }\OperatorTok{+}
\StringTok{  }\KeywordTok{xlab}\NormalTok{(}\StringTok{'Day of week'}\NormalTok{) }\OperatorTok{+}\StringTok{ }
\StringTok{  }\KeywordTok{ylab}\NormalTok{(}\StringTok{'Avg Events'}\NormalTok{) }\OperatorTok{+}
\StringTok{  }\KeywordTok{ggtitle}\NormalTok{(}\StringTok{'Average number of events for each Day of Week for each brand'}\NormalTok{)}
\end{Highlighting}
\end{Shaded}

\includegraphics{PebblePostProblemAnswer_files/figure-latex/Q2-1.pdf}

\subsection{Question 3}\label{question-3}

\paragraph{If you have any interesting observations about the sample
data, please describe them
here.}\label{if-you-have-any-interesting-observations-about-the-sample-data-please-describe-them-here.}

\begin{Shaded}
\begin{Highlighting}[]
\KeywordTok{str}\NormalTok{(event)}
\end{Highlighting}
\end{Shaded}

\begin{verbatim}
## 'data.frame':    1389990 obs. of  5 variables:
##  $ brand_id     : chr  "1472" "1472" "1472" "1472" ...
##  $ ppid         : chr  "35cd25e7-0edd-4094-8933-fe69adfc8dd1" "c6572797-d795-4d87-8964-8eb0bd0b01cf" "fcba9310-ff29-4576-bb99-c2cd800e63f6" "a54cbdc6-85ab-47bc-a0e2-aae928ed115c" ...
##  $ event_type   : chr  "seg" "seg" "seg" "seg" ...
##  $ device_family: chr  "Desktop" "Desktop" "Desktop" "Phone" ...
##  $ date         : Date, format: "2018-03-20" "2018-03-20" ...
\end{verbatim}

\begin{Shaded}
\begin{Highlighting}[]
\CommentTok{# unique values in event_type & device_family}
\KeywordTok{unique}\NormalTok{(event}\OperatorTok{$}\NormalTok{event_type)}
\end{Highlighting}
\end{Shaded}

\begin{verbatim}
## [1] "seg"  "conv"
\end{verbatim}

\begin{Shaded}
\begin{Highlighting}[]
\KeywordTok{unique}\NormalTok{(event}\OperatorTok{$}\NormalTok{device_family)}
\end{Highlighting}
\end{Shaded}

\begin{verbatim}
## [1] "Desktop" "Phone"   "Tablet"  "Other"   ""
\end{verbatim}

\begin{Shaded}
\begin{Highlighting}[]
\CommentTok{# Total events of each brand}
\KeywordTok{table}\NormalTok{(event}\OperatorTok{$}\NormalTok{brand_id)}
\end{Highlighting}
\end{Shaded}

\begin{verbatim}
## 
##    1034    1101    1472 
##   21256  187634 1181100
\end{verbatim}

\begin{Shaded}
\begin{Highlighting}[]
\CommentTok{# Total events of each type}
\KeywordTok{table}\NormalTok{(event}\OperatorTok{$}\NormalTok{event_type)}
\end{Highlighting}
\end{Shaded}

\begin{verbatim}
## 
##    conv     seg 
##    3214 1386776
\end{verbatim}

\begin{Shaded}
\begin{Highlighting}[]
\CommentTok{# Total events of each device family}
\KeywordTok{table}\NormalTok{(event}\OperatorTok{$}\NormalTok{device_family)}
\end{Highlighting}
\end{Shaded}

\begin{verbatim}
## 
##         Desktop   Other   Phone  Tablet 
##       2 1200507     806  147638   41037
\end{verbatim}

Note that there are duplicates both in event and cookie\_match df, I
will use ppid ``78C1840A54EE7F57EE1622290236DC03'' as an example.
Possible reason is the customer visit the web mutiple times per day.

\begin{Shaded}
\begin{Highlighting}[]
\CommentTok{# duplicates in event}
\NormalTok{event }\OperatorTok\StringTok{ }\KeywordTok{filter}\NormalTok{(ppid }\OperatorTok{==}\StringTok{ '78C1840A54EE7F57EE1622290236DC03'}\NormalTok{) }\OperatorTok\StringTok{ }\KeywordTok{kable}\NormalTok{()}
\end{Highlighting}
\end{Shaded}

\begin{longtable}[]{@{}lllll@{}}
\toprule
brand\_id & ppid & event\_type & device\_family & date\tabularnewline
\midrule
\endhead
1472 & 78C1840A54EE7F57EE1622290236DC03 & seg & Desktop &
2018-03-20\tabularnewline
1472 & 78C1840A54EE7F57EE1622290236DC03 & seg & Desktop &
2018-03-20\tabularnewline
1472 & 78C1840A54EE7F57EE1622290236DC03 & seg & Desktop &
2018-03-20\tabularnewline
1472 & 78C1840A54EE7F57EE1622290236DC03 & seg & Desktop &
2018-03-20\tabularnewline
1472 & 78C1840A54EE7F57EE1622290236DC03 & seg & Desktop &
2018-03-19\tabularnewline
1472 & 78C1840A54EE7F57EE1622290236DC03 & seg & Desktop &
2018-03-19\tabularnewline
1472 & 78C1840A54EE7F57EE1622290236DC03 & seg & Desktop &
2018-03-19\tabularnewline
1472 & 78C1840A54EE7F57EE1622290236DC03 & seg & Desktop &
2018-03-20\tabularnewline
1472 & 78C1840A54EE7F57EE1622290236DC03 & seg & Desktop &
2018-03-22\tabularnewline
1472 & 78C1840A54EE7F57EE1622290236DC03 & seg & Desktop &
2018-03-20\tabularnewline
1472 & 78C1840A54EE7F57EE1622290236DC03 & seg & Desktop &
2018-03-22\tabularnewline
1472 & 78C1840A54EE7F57EE1622290236DC03 & seg & Desktop &
2018-03-23\tabularnewline
1472 & 78C1840A54EE7F57EE1622290236DC03 & seg & Desktop &
2018-03-23\tabularnewline
1472 & 78C1840A54EE7F57EE1622290236DC03 & seg & Desktop &
2018-03-23\tabularnewline
1472 & 78C1840A54EE7F57EE1622290236DC03 & seg & Desktop &
2018-03-23\tabularnewline
1472 & 78C1840A54EE7F57EE1622290236DC03 & seg & Desktop &
2018-03-23\tabularnewline
1472 & 78C1840A54EE7F57EE1622290236DC03 & seg & Desktop &
2018-03-22\tabularnewline
1472 & 78C1840A54EE7F57EE1622290236DC03 & seg & Desktop &
2018-03-23\tabularnewline
1472 & 78C1840A54EE7F57EE1622290236DC03 & seg & Desktop &
2018-03-23\tabularnewline
1472 & 78C1840A54EE7F57EE1622290236DC03 & seg & Desktop &
2018-03-23\tabularnewline
1472 & 78C1840A54EE7F57EE1622290236DC03 & seg & Desktop &
2018-03-22\tabularnewline
1472 & 78C1840A54EE7F57EE1622290236DC03 & seg & Desktop &
2018-03-22\tabularnewline
1472 & 78C1840A54EE7F57EE1622290236DC03 & seg & Desktop &
2018-03-23\tabularnewline
1472 & 78C1840A54EE7F57EE1622290236DC03 & seg & Desktop &
2018-03-23\tabularnewline
1472 & 78C1840A54EE7F57EE1622290236DC03 & seg & Desktop &
2018-03-22\tabularnewline
1472 & 78C1840A54EE7F57EE1622290236DC03 & seg & Desktop &
2018-03-22\tabularnewline
1472 & 78C1840A54EE7F57EE1622290236DC03 & seg & Desktop &
2018-03-22\tabularnewline
1472 & 78C1840A54EE7F57EE1622290236DC03 & seg & Desktop &
2018-03-22\tabularnewline
1472 & 78C1840A54EE7F57EE1622290236DC03 & seg & Desktop &
2018-03-22\tabularnewline
1472 & 78C1840A54EE7F57EE1622290236DC03 & seg & Desktop &
2018-03-22\tabularnewline
1472 & 78C1840A54EE7F57EE1622290236DC03 & seg & Desktop &
2018-03-23\tabularnewline
1472 & 78C1840A54EE7F57EE1622290236DC03 & seg & Desktop &
2018-03-21\tabularnewline
1472 & 78C1840A54EE7F57EE1622290236DC03 & seg & Desktop &
2018-03-15\tabularnewline
\bottomrule
\end{longtable}

\begin{Shaded}
\begin{Highlighting}[]
\CommentTok{# duplicateds in cookie_match}
\NormalTok{cookie_match }\OperatorTok\StringTok{ }\KeywordTok{filter}\NormalTok{(ppid }\OperatorTok{==}\StringTok{ '78C1840A54EE7F57EE1622290236DC03'}\NormalTok{) }\OperatorTok\StringTok{ }\KeywordTok{kable}\NormalTok{()}
\end{Highlighting}
\end{Shaded}

\begin{longtable}[]{@{}lll@{}}
\toprule
ppid & matched\_id & date\tabularnewline
\midrule
\endhead
78C1840A54EE7F57EE1622290236DC03 &
AB7AE8BE80B83DC361D6AD726140E9726BC970BC & 2018-03-20\tabularnewline
78C1840A54EE7F57EE1622290236DC03 &
AB7AE8BE80B83DC361D6AD726140E9726BC970BC & 2018-03-22\tabularnewline
78C1840A54EE7F57EE1622290236DC03 &
AB7AE8BE80B83DC361D6AD726140E9726BC970BC & 2018-03-23\tabularnewline
78C1840A54EE7F57EE1622290236DC03 &
AB7AE8BE80B83DC361D6AD726140E9726BC970BC & 2018-03-20\tabularnewline
\bottomrule
\end{longtable}

\subsection{Question 4}\label{question-4}

\paragraph{How do you test the difference between the conversion rates
for test group and control group is statistically significant or
not?}\label{how-do-you-test-the-difference-between-the-conversion-rates-for-test-group-and-control-group-is-statistically-significant-or-not}

\begin{Shaded}
\begin{Highlighting}[]
\CommentTok{# create contingency table}
\NormalTok{mat <-}\StringTok{ }\KeywordTok{matrix}\NormalTok{(}\KeywordTok{c}\NormalTok{(}\DecValTok{500}\NormalTok{, }\DecValTok{200}\NormalTok{, }\DecValTok{10000}\NormalTok{, }\DecValTok{5000}\NormalTok{), }\DataTypeTok{nrow =} \DecValTok{2}\NormalTok{, }
             \DataTypeTok{dimnames =} \KeywordTok{list}\NormalTok{(}\KeywordTok{c}\NormalTok{(}\StringTok{'test'}\NormalTok{, }\StringTok{'control'}\NormalTok{), }
                             \KeywordTok{c}\NormalTok{(}\StringTok{'converted'}\NormalTok{, }\StringTok{'not_converted'}\NormalTok{)))}
\KeywordTok{chisq.test}\NormalTok{(mat)}
\end{Highlighting}
\end{Shaded}

\begin{verbatim}
## 
##  Pearson's Chi-squared test with Yates' continuity correction
## 
## data:  mat
## X-squared = 6.633, df = 1, p-value = 0.01001
\end{verbatim}

p-value = 0.01001, reject null hypothesis. It gives the strong evidence
to suggest that test and control group have statistical significance
difference.

\paragraph{What if the test group has 10000 users and 2 converters, and
the control group has 4000 users and 1
converter?}\label{what-if-the-test-group-has-10000-users-and-2-converters-and-the-control-group-has-4000-users-and-1-converter}

Sample size is too small to be statistically significant.

\subsection{Question 5}\label{question-5}

\begin{enumerate}
\def\labelenumi{\arabic{enumi}.}
\tightlist
\item
  How will you formulate the problem?
\end{enumerate}

Here, the outcome(converted, not\_converted) that we want to predict is
binary categorical outcome, and the label variable is given. So, this is
a supervise learning classification problem. I feel this analysis is
sensitive to date, I may consider to obtain the day of week,
seasonality, holiday as well. In other words, this may consider as a
time series problem as well.

\begin{enumerate}
\def\labelenumi{\arabic{enumi}.}
\setcounter{enumi}{1}
\tightlist
\item
  What users will you use as training and testing examples?
\end{enumerate}

For this two week data, I will use first 10 days data as the training
dataset, the rest as testing.

\begin{enumerate}
\def\labelenumi{\arabic{enumi}.}
\setcounter{enumi}{2}
\tightlist
\item
  What user features/data do you plan to collect?

  \begin{itemize}
  \tightlist
  \item
    Demographic data like age, gender, income, education, employment and
    etc.
  \item
    Social media data like facebook, twitter, and etc.
  \item
    Conversion history.
  \item
    Day and time when the user access the brand web.
  \item
    Device information.
  \item
    Number of times that the user visited the brand web.
  \end{itemize}
\item
  How will you preprocess the collected data to generate input for your
  system?
\end{enumerate}

Basiclly, cleaning, transforming. Not going to go deep in cleaning,
because it may vary depends on different data. For transforming,

\begin{itemize}
\tightlist
\item
  Create variables. For example, I may create zipcode group if user
  zipcode is available. This step is based on the marketing expertise
  suggestion as well as each sub-group contains at least enough events(I
  prefer 10 cases) to build the model.
\item
  Create dummy variables.
\item
  Imputation variables.
\item
  Scaleing and centering.
\item
  Run PCA/MCA analysis to try to get insights of the features, reduce
  the dimensions of the features. This step include remove correlated
  features, remove zero/near-zero variance features.
\end{itemize}

\begin{enumerate}
\def\labelenumi{\arabic{enumi}.}
\setcounter{enumi}{4}
\tightlist
\item
  What algorithm(s) to use and why?
\end{enumerate}

All the algorithms that I pick will friendly for binary category
outcome, mixed feature types supervise learning classification problem.
I will try use elastic regression first, since the training time was the
fastest one. Then I can get an general view of the model. Then will try
use logistic regression, random forest, svm, and etc.

\begin{enumerate}
\def\labelenumi{\arabic{enumi}.}
\setcounter{enumi}{5}
\tightlist
\item
  How will you evaluate the performance?
\end{enumerate}

Cross-validation while tring the model and use AUC, confusion
matrix(accuracy, sensitivity, specificity) to evaluate the model as well
as evaluate the performance while using the test dataset.

\subsection{Question 6}\label{question-6}

Continuous from built model from Question 5, the uplift modeling's
general Steps:

\begin{enumerate}
\def\labelenumi{\arabic{enumi}.}
\tightlist
\item
  Predict the outcome on the promotional item applied users.
\item
  Predict the outcome on the no promotional item applied users.
\item
  Find the uplift as the difference in the rates (step 1 - step 2).
\item
  Find upper and lower confidence limits on the uplift.
\end{enumerate}

Results:

\begin{itemize}
\tightlist
\item
  If confidence limits of the uplift includes zero. The promotion effect
  is unknow and not significant.
\item
  If confidence limits of the uplift significantly greater than zero,
  those are swing user.
\item
  If confidience limits of the uplift significantly less than zero,
  those are the no purchase user.
\end{itemize}

The uplift package in R can handle this type of modeling.


\end{document}
